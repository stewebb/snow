\documentclass{article}

% ready for submission
\usepackage[final]{neurips_2023}
\usepackage{amsmath}
\usepackage{graphicx}
\usepackage{float}
\usepackage{subfig}         % display sub-figure
\usepackage[utf8]{inputenc} % allow utf-8 input
\usepackage[T1]{fontenc}    % use 8-bit T1 fonts
\usepackage{hyperref}       % hyperlinks
\usepackage{url}            % simple URL typesetting
\usepackage{booktabs}       % professional-quality tables
\usepackage{amsfonts}       % blackboard math symbols
\usepackage{nicefrac}       % compact symbols for 1/2, etc.
\usepackage{microtype}      % microtypography
\usepackage{xcolor}         % colors

\title{Dynamic Daytime Simulation with Snow Effect}

\author{%
  Steven Webb\\\\
  u7544998 
   \And
  Leosha Trushin\\\\
  u7302755\\
  \AND  The Australian National University 
}

\begin{document}

\maketitle

\begin{abstract}
TODO:
\end{abstract}

\section{Introduction and Background}

\section{Motivations and Aims}

\section {Methods and Algorithms}

% Method part 1: Snow Rendering
\subsection {Snow Rendering}

% Snow method 1: Color
\subsubsection {Snow Color Function}
The snow color is defined by a modified Phong reflection model, where the diffuse color has been set to nearly white and the surface 
normal has been distorted for a more realistic snowing effect.
\[
  C_{s} = k_a \cdot I_a + \sum_{m \in \text{lights}} (k_d \cdot (L_m \cdot (N + \alpha n)) + k_s \cdot (R_m \cdot V)^\beta) \cdot I_m  
\]

where:
\begin{itemize}
  \item \( C_{s} \) is the calculated snow color.
  \item \( k_a \) is the ambient reflection coefficient. In this paper, \( k_a = (0.05, 0.05, 0.05)\)
  \item \( I_a \) is the ambient light intensity, \( I_a \in [0, 1]\)
  \item \( k_d \) is the diffuse reflection coefficient, which is the snow color. To make a more glittering effect, % TODO: REF
  the blue component of it can be slightly higher. In this paper, \( k_d = (0.9375, 0.9375, 1)\)
  \item \( k_s \) is the specular reflection coefficient. In this paper, \( k_s = (0.2, 0.2, 0.2)\)
  \item \( L_m \) is a vector from the point to the light source \( m \). It must be normalized.
  \item \( N \) is the surface normal (in camera space) of the point. It must be normalized.
  \item \( \alpha \) is the distortion coefficient. In this paper, \( \alpha = 0.1\)
  \item \( n \) is a random vector, together with \( \alpha \) to control the distortion of the surface normal. Each element of this 
  vector \(\in [0, 1]\)
  \item \( R_m \) is the direction of the reflected light. It must be normalized.
  \item \( V \) is a vector from the surface point to the eye/camera position. It must be normalized.
  \item \( \beta \) is the specular exponent to control the shininess. It should be a small value as snow is not very shiny. In this 
  paper, \( \beta = 25\)
  \item \( I_m \) is the intensity of light source \( m \), \( I_m \in [0, 1]\)
\end{itemize}

% Snow method 2: Accumulation Prediction Coefficient
\subsubsection {Snow Accumulation Prediction Coefficient}
Assume there is no wind in the scene and all snow falls vertically. The snow accumulation prediction coefficient 
\( f_{s} \) on a point can be calculated by the multiplication of an exposure value \( f_{e} \), an inclination 
value \( f_{inc} \), and a user-defined value \( f_{u} \).
\[
  f_{s} = f_{e} \cdot f_{inc} \cdot f_{u}
\]

The exposure value \( f_{e} \) indicates occlusions that prevent snow from falling on the surface. % TODO: REF
\( f_{e} \in [0, 1] \). \( f_{e}=0 \) means that the point is fully occluded by another object, which is right 
above it, so there are no snowing effects on this point. \( f_{e}=1 \) means that the point is fully exposed to the 
falling snow. This value is based on a shadow map by setting a virtual directional light right above the object. 
The light direction is to the ground, so the points in the virtual shadow are also occluded by another object. 
Besides, to make the transition process smoother, soft shadowing techniques can be used. 

figure 1 is the occlusion map, which uses different colors to indicate the exposure value. Low hue 
values (e.g., red) stand for a low exposure value (i.e., the point is occluded by another object), while high hue 
values (e.g., purple) stand for a high exposure value.

\begin{figure}[h]
  \centering
  \begin{minipage}{0.4\textwidth}
      \centering
      \includegraphics[width=\textwidth]{images/OcclusionMap.png}
      \caption{Occlusion Map}
      \label{fig:OcclusionMap}
  \end{minipage}\hfill
  \begin{minipage}{0.4\textwidth}
      \centering
      \includegraphics[width=\textwidth]{images/InclinationMap.png}
      \caption{Inclination Map}
      \label{fig:InclinationMap}
  \end{minipage}
\end{figure}

The inclination value \( f_{inc} \) depends on the angle between the surface normal in world space and the horizon 
\( f_{inc} \in [0, 1] \). \( f_{inc}=1 \) means that the point is on a flat surface that is parallel to the horizon. 
\( f_{inc}=0 \) the point is on a surface whose normal is perpendicular to the horizon. This value can be calculated by:
\[
  f_{inc}=
  \left\{
    \begin{array}{ll}
      N \cdot U + n & N \cdot U > 0 \\
      0 & N \cdot U \leq 0 \\
    \end{array} 
  \right. 
\]

where:
\begin{itemize}
  \item \( N \) is the surface normal of the target point (normalized).
  \item \( U \) is the surface normal of the horizon (normalized). It is a constant vector \((0, 0, 1)\)
  \item \( N \cdot U \) is the cosine of the angle between the above two surface normals, ranging from -1 to 1. If the angle is less 
  than 90$^{\circ}$, \( N \cdot U > 0\).
  \item \( n \) is a small random number to make snow have a more natural look. In this paper, \(n = rand(0, 0.4)\).
\end{itemize}

Figure 2 is the inclination map, which also uses different colors to indicate the exposure value. Low hue values (e.g., red) stand for 
a low inclination value (i.e., the angle between two surface normals is big), while high hue values (e.g., purple) stand for a high 
inclination value.

The user-defined function \( f_{u} \) is used to further customize and manipulate the snow amount \( f_{u} \in [0, 1] \). For example, 
\( f_{u}=0 \) means manually disabling the snow effect. This function is crucial in the daytime simulation.

% Snow method 3: Full Snow Equation
\subsubsection {Full Snow Equation}
The final color is a blending of the snow color and the surface color (without snow) on that point, by using the snow accumulation 
prediction coefficient in the last sub-section. This is also called the "Full Snow Equation".
\[
  C = f_{s} * C_{s} + (1-f_{s}) * C_{o}
\]

where:
\begin{itemize}
  \item \( C \) is the final color.
  \item \( f_{s} \) is the snow accumulation prediction coefficient.
  \item \( C_{s} \) is the snow color.
  \item \( C_{o} \) is the surface color without snow.
\end{itemize}

% Method part 2: Environment Simulation
\subsection {Environment Simulation}

%  Environment 1: Temperature
\subsubsection {Temperature}
The temperature of a location depends on multiple factors, e.g., time of the day, season, altitude, latitude and distance to the ocean. 
To create a reasonable snow effect, the location used in this paper is an in-land city around the latitude of 35$^{\circ}$ N or 35
$^{\circ}$ S and the season is winter.

This location has a large temperature difference. The highest temperature is about \(10^\circ\mathrm{C}\), which can be reached at 
around 1-2 P.M. The lowest temperature is about \(-10^\circ\mathrm{C}\), which can be reached at around 5-7 A.M. The hourly 
temperature data can be obtained on weather forecast websites, and the Cubic Spline method is used to interpolate the intermediate 
values, making the temperature-time curve smoother.

%  Environment 2: Snow Factor
\subsubsection {Snow Factor}
The snow factor is the user-defined function \( f_{u} \) in Section 4.1, causing a potentially negative effect on the original snow. 
The range is between 0 and 1, where 0 means that the snow is disabled regardless of the original snow, and 1 means that there are no 
negative effects. To simplify the simulation, the snow factor is based solely on the environmental temperature. 

If the temperature is below or equal to \(0^\circ\mathrm{C}\), the snow factor is always 1. If the temperature is above or equal to 
\(5^\circ\mathrm{C}\), the snow factor is always 0. Otherwise, the snow factor decreases linearly from 1 to 0 as temperature increases.
\[
  f_{u}(p)=
  \left\{
    \begin{array}{ll}
      0 & t\leq 0 \\
      1 - \frac{t}{5} &  0 < t < 5 \\
      1 & t\geq 5 \\
    \end{array} 
  \right. 
\]
\text{where } t \text{ is the temperature in degrees Celsius.}

% Environment 3: Sunlight Intensity and Direction
\subsubsection {Sunlight Intensity and Direction}
In the real world, there are multiple light sources outside, including the sun, sky, moon, stars, and human-made lights. In this paper, 
only the sunlight and sky (ambient) light are considered. The intensity of the skylight is a small constant value while that of the 
sunlight varies by time, ranging from 0 to 1.

Assume the light intensity of the sun is 0 (or a very small value) at night. It starts increasing rapidly (typically 30 minutes) 
before sunrise and reaches the maximum value in a short time (typically 1-3 hours). It remains stable until several hours before the 
sunset. The value decreases quickly in the sunset period and finally returns to 0 (or a very small value). Here is the mathematical 
formula of the light intensity.
\[
  I = e^{-\frac{\left(\frac{\pi}{2} - \theta\right)^{\frac{m}{60} + 1}}{b}}
\]

where:
\begin{itemize}
  \item \( I \) is the light intensity of the simulated sun, \( I \in (0, 1] \).
  \item \( \theta \) is the solar elevation angle in radians, which is based on the current time, season (Earth declination) and 
  position (latitude).
  \item \( m \) is the total daylight time (in minutes) in that day, \( m \in [0, 1440) \), \( m \in \mathbb{Z} \). For example, if 
  the sunrise time is 6 AM and the sunset time is 6 PM, \( m=720 \) (12 hours)
  \item \( b \) is a bias factor adjusting the exponent decay rate, making the light intensity transition more realistic.
\end{itemize}

The bias \( b \) depends on multiple factors, including the altitude, latitude and time zone. To simplify the simulation, it can be 
calculated by solving an equation in terms of \( m \) when 
\( I \) and \( \theta\) is given.
\[
  b = -\frac{\left(\frac{\pi}{2} - \theta\right)^{\frac{m}{60} + 1}}{\log(I)}
\]

Assume the solar light intensity is 0.1 on sunrise or sunset (i.e., \( \theta=0\)), the formula of the bias can be expressed as:
\[
  b = \frac{\left(\frac{\pi}{2}\right)^{\frac{m}{60} + 1}}{\log(10)}
\]

By giving the sun elevation angle \( \theta \) and azimuth angle \( \phi \), the sunlight direction can be calculated by:
\[
  (-sin(\phi)cos(\theta), -cos(\phi)cos(\theta), sin(\theta))
\]

%  Environment 4: Sunlight and Sky Color
\subsubsection {Sunlight and Sky Color}
Sunlight and sky colors depend on the current time. Typically on a sunny day, the sunlight is yellow to white for most of the day and 
it becomes orange during sunrise or sunset. The sky is light blue in the day, orange during sunrise or sunset and dark purple at night. 
To simulate those colors, interpolation is also needed to make the transition process smoother. 
\[
  C_{b}=
  \left\{
    \begin{array}{ll}
      \text{night} & \theta \leq 10 \\
      \text{interpolate(night, twilight)} &  -10 \leq t < 5 \\
      \text{interpolate(twilight, day)} &  5 \leq t < 20 \\
      \text{day} & t \geq 20 \\
    \end{array} 
  \right. 
\]

\section{Experiments and Results}

% Result 1: Pure Snow Effect
\subsection {Pure Snow Effect}
Figure 3 is the rendering effect with full snow (a), half snow (b), and no snow (c) respectively. The dynamic daylight simulation is 
disabled and the light source is always right above the object with the maximum intensity and white color. Those three sub-figures 
illustrate the effect on both declination (the head and book part of the statue) and occlusion (the feet area of it), as well as the 
color blending mentioned above.

\begin{figure}[h]
  \centering
  \subfloat[Snow Amount = 1.0]{
    \includegraphics[width=0.33\textwidth]{images/T9999M10P.png}
    \label{fig:T9999M10P}
  }
  \subfloat[Snow Amount = 0.5]{
    \includegraphics[width=0.33\textwidth]{images/T9999M05P.png}
    \label{fig:T9999M05P}
  }
  \subfloat[Snow Amount = 0.0]{
    \includegraphics[width=0.33\textwidth]{images/T9999M00P.png}
    \label{fig:T9999M00P}
  }
  \caption{Object covered by different amounts of snow}
  \label{fig:PureSnow}
\end{figure}

% Result 2: Snow Effect with Dynamic Daylight Simulation (Medium Latitude)
\subsection {Snow Effect with Dynamic Daylight Simulation (Medium Latitude)}

Figure 4 is the plot of the data at different times with a medium latitude location on the winter solstice (35 $^{\circ}$S on June 
\(22^{nd}\)). The sunrise time is about 7:00 AM and the sunset time is about 5:00 PM, so there are 10 hours of daylight time.

\begin{figure}[h]
  \centering
  \begin{minipage}{1.00\textwidth}
      \centering
      \includegraphics[width=\textwidth]{images/Plot35N.png}
      \caption{Relevant data in 35 $^{\circ}$S on June \(22^{nd}\)}
      \label{fig:Plot35N}
  \end{minipage}
\end{figure}

Figure 5 shows the rendering effect at different times in the above location. Throughout the night (e.g., midnight (a) or 7:00 PM (f)), 
the temperature is quite low and the sunlight intensity is almost zero so the object is fully covered by white snow and only ambient 
colors can be seen. The sky color should be very dark.

In the morning (e.g., 8:00 AM (b)), the sunlight intensity increases with an orange color, resulting in a yellow snow color and an 
orange sky color. The snow coverage is still 100 \% as the temperature is still below \(0^\circ\mathrm{C}\). Around noon (e.g., 11:00 
AM (c)), the sky becomes light blue and the snow color becomes back to white as the sunlight color is pretty white. The snow starts 
melting at this time so part of the covered object can be seen now. 

In the afternoon (e.g., 2:00 PM (d)), the weather is warmest. As a result, there is no snow on the scene. During the sunset period, 
(e.g., 5:00 PM (e)), the sunlight intensity drops back and the sunlight color/sky color becomes orange again. The snow effect is back 
due to the dropping temperature.

\begin{figure}[h]
  \centering
  \subfloat[Time = 12:00 AM]{
    \includegraphics[width=0.33\textwidth]{images/T0000L35S.png}
    \label{fig:T0000L35S}
  }
  \subfloat[Time = 08:00 AM]{
    \includegraphics[width=0.33\textwidth]{images/T0800L35S.png}
    \label{fig:T0800L35S}
  }
  \subfloat[Time = 11:00 AM]{
    \includegraphics[width=0.33\textwidth]{images/T1100L35S.png}
    \label{fig:T1100L35S}
  }

  \subfloat[Time = 02:00 PM]{
    \includegraphics[width=0.33\textwidth]{images/T1400L35S.png}
    \label{fig:T1400L35S}
  }
  \subfloat[Time = 05:00 PM]{
    \includegraphics[width=0.33\textwidth]{images/T1700L35S.png}
    \label{fig:T1700L35S}
  }
  \subfloat[Time = 07:00 PM]{
    \includegraphics[width=0.33\textwidth]{images/T1900L35S.png}
    \label{fig:T1900L35S}
  }

  \caption{The rendered scene in different time (at 35 $^{\circ}$S on June \(22^{nd}\))}
  \label{fig:L35S}
\end{figure}

% Result 3: Snow Effect with Dynamic Daylight Simulation (High Latitude)
\subsection {Snow Effect with Dynamic Daylight Simulation (High Latitude)}

Figure 6 is the plot of the data at different times with a high latitude location on the summer solstice (60 $^{\circ}$N on June 
\(22^{nd}\)). The sunrise time is about 2:30 AM and the sunset time is about 9:30 PM, so there are 19 hours of daylight time.

\begin{figure}[h]
  \centering
  \begin{minipage}{1.00\textwidth}
      \centering
      \includegraphics[width=\textwidth]{images/Plot60S.png}
      \caption{Relevant data in 60 $^{\circ}$N on June \(22^{nd}\)}
      \label{fig:Plot60S}
  \end{minipage}
\end{figure}

Figure 7 is the rendering effect in the above location. In the summer of the high-latitude locations, each day has a long daylight 
time. Throughout the night (e.g., midnight (a)), the sky is not completely dark even if the sun is below the horizon. 

At 8:00 AM (b), the sunlight intensity has already reached the maximum value. The scenes between the two locations are similar during 
the noon or afternoon time (c or d), except for the solar position. However, because of the high latitude, the sunlight intensity 
keeps the high value at 5:00 PM (e) and finally starts dropping at around 7:00 PM (f).

\begin{figure}[h]
  \centering
  \subfloat[Time = 12:00 AM]{
    \includegraphics[width=0.3\textwidth]{images/T0000L60N.png}
    \label{fig:T0000L60N}
  }\hfill
  \subfloat[Time = 08:00 AM]{
    \includegraphics[width=0.3\textwidth]{images/T0800L60N.png}
    \label{fig:T0800L60N}
  }\hfill
  \subfloat[Time = 11:00 AM]{
    \includegraphics[width=0.3\textwidth]{images/T1100L60N.png}
    \label{fig:T1100L60N}
  }
  
  \subfloat[Time = 02:00 PM]{
    \includegraphics[width=0.3\textwidth]{images/T1400L60N.png}
    \label{fig:T1400L60N}
  }\hfill
  \subfloat[Time = 05:00 PM]{
    \includegraphics[width=0.3\textwidth]{images/T1700L60N.png}
    \label{fig:T1700L60N}
  }\hfill
  \subfloat[Time = 07:00 PM]{
    \includegraphics[width=0.3\textwidth]{images/T1900L60N.png}
    \label{fig:T1900L60N}
  }

  \caption{The rendered scene in different time (at 60 $^{\circ}$N on June \(22^{nd}\))}
  \label{fig:L60N}
\end{figure}

\section{Discussions}

% Discussion 1: Temperature and Snow Factor Simulation
\subsection {Temperature Simulation}
In this paper, the temperature curve throughout the day is based on the real meteorological data of a location with the Cubic Spline 
interpolation and there is no mapping between the location and the temperature. All the temperature data are manually defined so some 
problems may be caused in this paper if the location is set to most of the equator area. However, this is not a bad method for 
simulating temperature in some applications like video games or movies, as there are only a few scenes that need to be considered and 
in some scenes the temperature curve is irregular (e.g., on a "Hot Jupiter" exoplanet).

\subsection {Temperature Snow factor simulation Simulation}
Snow factor simulation has similar problems. This paper only implements a simple linear mapping on temperature and the snow factor. But
in the real world, the cause of snow is very complicated. However, this is still a simple approach to get started on the environment
simulation. 

% Discussion 1: Complex Scenes
\subsection {Sunlight and Sky}

% Discussion 1: Complex Scenes
\subsection {Thickness of the Snow}

% Discussion 1: Complex Scenes
\subsection {Complex Scenes}

% Discussion 1: Complex Scenes
\subsection {Performance}

\section{Conclusion and Future Work}
Please prepare submission files with paper size ``US Letter,'' and not, for
example, ``A4.''

Fonts were the main cause of problems in the past years. Your PDF file must only
contain Type 1 or Embedded TrueType fonts. Here are a few instructions to
achieve this.


\section*{References}

{
\small


[1] Alexander, J.A.\ \& Mozer, M.C.\ (1995) Template-based algorithms for
connectionist rule extraction. In G.\ Tesauro, D.S.\ Touretzky and T.K.\ Leen
(eds.), {\it Advances in Neural Information Processing Systems 7},
pp.\ 609--616. Cambridge, MA: MIT Press.


[2] Bower, J.M.\ \& Beeman, D.\ (1995) {\it The Book of GENESIS: Exploring
  Realistic Neural Models with the General Neural Simulation System.}  New York:
TELOS/Springer--Verlag.


[3] Hasselmo, M.E., Schnell, E.\ \& Barkai, E.\ (1995) Dynamics of learning and
recall at excitatory recurrent synapses and cholinergic modulation in rat
hippocampal region CA3. {\it Journal of Neuroscience} {\bf 15}(7):5249-5262.
}

\section*{Additional Experiment Results}

\section*{Confidential Peer Review} 

The percentage, who did what, ratio weights. 






\end{document}